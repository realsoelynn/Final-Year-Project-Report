\chapter{Introduction}
Traditionally, FPGAs is programmed using manual RTL design. Although RTL design are complicated and it is an inefficient process. InTime \citep{kapre2015intime} try to solves this problem by coupling the compute power of cloud-based technologies and machine learning.   

Open Computing Language (OpenCL) \citep{opencl} is a framework for platform-independent parallel programming across CPUs, GPUS and other discrete computing devices. OpenCL-based High Level Synthesis (HLS) frameworks is becoming mainstream for programming Field-Programmable Gate Arrays (FPGAs) with the active support from major FPGAs vendors. \citep{rich2015how}

The goal of this project is to develop a machine learning program to predict the optimization of the OpenCL-based HLS for FPGAs which in term will improve the performance of the program. The proposed program will streamline the machine learning on the OpenCL-based HLS data and reduce the time required to compile the program.



The key contributions of this paper include:
\begin{itemize}
    \item Performance comparison between various machine learning algorithms for classification of seven output metrics from Quartus OpenCL compiler.
    \item Python program that streamlines machine learning on Quartus OpenCL compiler output.
\end{itemize}
