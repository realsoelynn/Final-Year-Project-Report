\chapter{Machine Learning}

\section{Algorithms}
In this project, we picked 8 type of classifiers. They are:
\begin{enumerate}
    \item RandomForest
        \begin{itemize}
            \item Random Forest classifier is a tree based model and it operates by constructing multiple decision trees during training. It corrects the overfitting problem of decision trees.
        \end{itemize}
    \item ExtraTrees
        \begin{itemize}
            \item  
            ExtraTress is an Extremely Randomized Trees. The difference between ExtraTrees and RandomForest is that ExtraTrees goes one step further in the splitting are carried out. In RandomForest, random group of features is used, but instead of filtering out by the thresholds, the thresholds are randomly drawn for each candidate feature and the best thresholds among the randomly-generated ones is picked as the splitting rule. But, this reduce the variance of the model with a slight increase in bias.
        \end{itemize}
    \item NaiveBayes
        \begin{itemize}
            \item Naive Bayes classifiers is a type of probabilistic classifiers based on Bayes' theorem. 
        \end{itemize}
    \item LogitBoosting
        \begin{itemize}
            \item LogitBoosting is an addictive, boosted model that tries to reduce logistic errors.
        \end{itemize}
    \item SVM
        \begin{itemize}
            \item Support vector machines (SVM) separates the positive class from the negative one by a separating hyperplane by maximizing distance between them. If data points are not directly separable, a higher dimension space would be used to transform the data points where such separation is possible.
        \end{itemize}
    \item Bagging
        \begin{itemize}
            \item Bagging stands for Bootstrap Aggregation. Bagging averages over the predictions of all models. It fits base classifiers on each random subsets of the original data set and aggregate the individual predictions to form a final prediction.
        \end{itemize}
    \item SGDClassifier
        \begin{itemize}
            \item SGDClassifier is a linear classifier with stochastic gradient descent (GD) learning. 
        \end{itemize}
    \item NeuralNetwork
        \begin{itemize}
            \item Neural Network is a classifier that mimics human brains. In this project, we have input layer and output layer. Input layer comprised of 5 neurons and a single neuron for the output layer.
        \end{itemize}
\end{enumerate}

\section{Program Flow}

The flow of the program is structured as follows:

\begin{enumerate}
    \item Converting numerical response into categorical data
        \begin{itemize}
            \item Classifiers required a class label to classify based on the input (features). Thus for all the seven output metrics, the numeric data are classified into binary class 0 or 1 with 1 representing the numeric data is greater than median. Otherwise, 0. 
        \end{itemize}
    \item Validating Models
        \begin{itemize}
            \item K-fold cross validation method is used to validate models and also to limit the overfitting problems.
        \end{itemize}
    \item Generate ROC curve and AUC charts
        \begin{itemize}
            \item ROC curve and AUC charts are then plotted to be used in evaluating the performance of the classifiers.
        \end{itemize}
\end{enumerate}

Overall program flow is as follows. For each fold of the training and test dataset, each model is fitted and then the various evaluation metrics such as TPR, FPR and AUC are saved into json file. After training and testing of models are done for all folds of data, ROC curve and AUC charts are plotted.